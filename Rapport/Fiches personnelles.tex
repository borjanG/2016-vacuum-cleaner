\documentclass[a4paper,12pt]{report}
\usepackage[utf8]{inputenc}

\begin{document}
\begin{center}
\section*{Fiche personnelle 1}
\end{center}
\ \ \\
\subsection*{Le contenu du cours a-t-il répondu à vos attentes ainsi qu’aux objectifs annoncés par l’intervenant ?}
\ \ \ \ \ De mon point de vue, l'objectif principal de l'initiation à l'IA est respecté ; mais face à cet objectif je m'attendais à plus de théorie. En effet, les principes théoriques du cours appliqués au projet m'ont moins intéressé qu'une théorie pure associé à des illustrations plus célèbres. J'aurais peut être préféré approcher les principes généraux, et nous intéresser aux travaux célèbres de l'histoire, aux applications du quotidien ... Pour les travaux dirigés peut-être qu'une étude de cas serait intéressante pour approcher la pratique. (Je conçois que cette approche soit peut-être plus fastidieuse pour l'enseignant).
\subsection*{Comment jugez-vous l’équilibre entre théorie et pratique ?}
\ \ \ \ \ Comme dit dans la question précédente, j'aurais aimé avoir plus de théorie que face à la quantité de pratique. En effet, il a été intéressant d'aborder l'IA en se positionnant au coeur du sujet, mais j'ai l'impression de qu'il me manque de théorie pour vraiment cerner le domaine de l'IA, et par conséquent avoir envie de poursuivre dans cette voie. De plus utiliser le sujet du projet comme application du cours ne nous permet pas de vraiment segmenter cours et pratique.
\subsection*{Que pensez-vous de l’approche pédagogique mise en oeuvre dans le cours ?}
\ \ \ \ \ Je pense qu'associer le projet aux cours d'amphi, n'est pas la meilleur approche. Il aurait été intéressant d'appliquer le cours à un autre exemple, et qu'il nous revienne de reproduire cette application à notre sujet de projet. \\
L'approche pédagogique en Travaux dirigés est un très bon mélange de liberté et de soutien. Le professeur est disponible pour répondre aux questions ou nous apporter de l'aide, tout en nous laissant travailler librement. De plus, il nous incite à faire la démarche pour demander de l'aide ou pour poser des questions, ce qui est pour moi quelque chose de bien pour la vie en générale.
\subsection*{Que pensez-vous du projet comme mode d’évaluation, comme consommateur de temps dans le cadre du cours ?}
\ \ \ \ \  Le projet est une bonne idée d'évaluation mais il est relativement couteux en temps. En effet, je pense que le niveau de programmation demandé ne correspond pas forcément au niveau que l'on a pu acquérir au cours de cette licence. Tous autant que nous sommes, nous nous sommes pris la tête pour arriver à passer les tests. J'ai pu constater que certains groupes avaient abandonné la partie programmation du projet face au niveau de difficulté et au temps que cela nous a demandé. \\
Je pense que ce projet est trop gourmand en temps. Le fait d'appliquer les concepts du cours au projet ne permet pas de dissocier les deux réellement, ce qui donne l'impression que le projet représente à lui seul l'UE.
\subsection*{Avez-vous des remarques à faire en vue de l’amélioration du cours ? Ou bien pensez-vous que ce cours devrait
disparaître de votre formation ?}
\ \ \ \ \ Pour répondre à cette question, je vais reprendre les grandes lignes de ce que j'ai pu énoncer précédemment. Je pense qu'une approche plus théorique comme décrite plus haut ferait que les élèves s'intéresseraient plus à ce module. De plus, de mon point de vue, diminuer le niveau de programmation attendu permettrait de plus s'intéresser au concept d'intelligence d'artificielle même, au lieu de se prendre la tête pendant des heures sur la partie programme. 

\subsection*{Éventuellement, si vous avez beaucoup souffert pour la partie informatique, donnez les raisons (autres que je
n’ai jamais aimé l’informatique) qui pourraient expliquer vos difficultés (parcours antérieurs, mise à niveau, . . .)}
\ \ \ \ \  Le niveau informatique demandé dans le projet est selon moi trop élevé pour la plupart des étudiants de la licence. Nous ne sommes pas dans une licence d'informatique pure, donc malgré quelques génies parmis nous, notre niveau n'a rien d'exceptionnel (chose dont seuls les professeurs de programmation semblent savoir).

\end{document}